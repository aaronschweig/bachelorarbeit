\documentclass[
	12pt,
	BCOR=5mm,
	DIV=12,
	headinclude=on,
	footinclude=off,
	parskip=half,
	bibliography=totoc,
	listof=entryprefix,
	toc=listof,
	numbers=noenddot,
	plainfootsepline
]{scrreprt}

%	Konfigurationsdatei einziehen
\input{config}

\begin{document}

\TitelDerArbeit{Entwurf und Implementierung einer Referenzarchitektur zur automatischen Erzeugung digitaler Zwillinge}
\AutorDerArbeit{Aaron Schweig}
\Firma{Hays AG}
\Kurs{WWI18SEC}

\input{titlepage}


% Ehrenwörtliche Erklärung ewerkl.tex einziehen
\input{ewerkl.tex}

\pagenumbering{roman} % Römische Seitennummerierung
\normalfont

%	Kurzfassung
\chapter*{Kurzfassung}
\begingroup
\begin{table}[h!]
\setlength\tabcolsep{0pt}
\begin{tabular}{p{3.7cm}p{11.7cm}}
Titel: & \DerTitelDerArbeit \\
Verfasser/in: & \DerAutorDerArbeit \\
Kurs: & \DieKursbezeichnung \\
Ausbildungsstätte: & \DerNameDerFirma\\
\end{tabular}
\end{table}
\endgroup

In folgender Arbeit wird die automatische Erzeugung digitaler Zwillinge thematisiert. Anhand der Defintion eines digitalen Zwillings werden Anforderungen abgeleitet, die ein System mit sich bringen muss, um digitale Zwillinge abzubilden. Dafür werden nachdem Charakteristika digitaler Zwillinge herausgearbeitet wurden, Probleme definiert, die einer automatischen Erstellung im Wege stehen. Zunächst werden Kernkomponentenin Form einer Device Registry und eines Digital Twin Providers identifiziert. Anschließend wird ein Lösungsvorschlag auf Basis der Herausforderungen erstellt, der in der Einführung einer weiteren Komponente resultiert. In einer ausführlichen Analyse werden die identifizierten Probleme bearbeitet, sodass klare funktionelle Anforderungen an die einzelnen Komponenten gestellt werden können. Nachdem eine Analyse der einzelnen Komponenten durchgeführt wurde, wird anhand einer prototypischen Umsetzung die Validität des entwickelten Konzepts überprüft.



%	Inhaltsverzeichnis
\tableofcontents

%	Abbildungsverzeichnis
\listoffigures

%	Tabellenverzeichnis
\listoftables

%	Listingverzeichnis
% \lstlistoflistings

% 	Abkürzungsverzeichnis (siehe Datei acronyms.tex!)
\clearpage
\chapter*{Abkürzungsverzeichnis}	
\addcontentsline{toc}{chapter}{Abkürzungsverzeichnis}


\begin{acronym}
	\acro{DHBW}{Duale Hochschule Baden-Württemberg}
	\acro{DT}{Digitaler Zwilling}
	\acro{PHM}{Prognostics \& Health Management}
\end{acronym}
\ohead{Acronyms} % Neue Header-Definition

%--------------------------------
% Start des Textteils der Arbeit
%--------------------------------
\clearpage
\ihead{\chaptername~\thechapter}
\ohead{\headmark}
\pagenumbering{arabic}

\begin{enumerate}
	\item Einleitung
	\item Grundlagen zu Digital Twins
	\begin{enumerate}
		\item Was sind DTs
		\begin{itemize}
			\item Definitionen \autocite[\ppno~108953]{fuller2020digital}
			\item Unterscheidung verschiedener Arten und Abgrenzung zum cyber-physical System
		\end{itemize}
		\item Eigenschaften
		\item Ökonomischer Nutzen
		\begin{itemize}
			\item Vorteile in verschiedenen Bereichen der Anwendung
			\item Industrie
			\item Product-Lifecycle
			\item Schaeffler Windanlagen?
			\item E-Technik
			\item Simulationen
			\item etc.
		\end{itemize}
	\end{enumerate}
	\item Problemstellung und Lösungsansatz
	\begin{enumerate}
		\item Automatische Erzeugung digitaler Zwillinge
		\item Ist-Zustand in der aktuellen Umgebung
		\item Soll-Zustand
		\item Delta zwischen Ist und Soll
	\end{enumerate}
	\item Konzeptentwicklung
	\begin{enumerate}
		\item Architekturkonzept
		\begin{itemize}
			\item Wieso Device-Registry? Anwendung, Anforderungen, Nutzen, etc.
			\item Authentifizierung, DaaS(Device as a Service), massenhafte Authentifizierung
			\item Common Schnittstellenformate, welche sind im Umlauf wie können sie zusammengeführt werden, sodass alle obigen Anforderungen noch laufen und erfüllt sind?
			\item Unterscheidung zwischen einer Device-Registry und der Speicherung als DT
			\item Was muss eine DT-Architektur mit sich bringen, damit Definition erfüllt Ist?
			\item Kombination zwischen Device-Registry und DT (hier liegt das Kernproblem!)
			\item verschiedene Zugriffsarten auf die Daten eines DT
			\item Welche Daten sind vorhanden und wie kann sichergestellt werden, dass ein speziell Modell für DT's vorliegt? $\rightarrow$ Hier kommt dan Vorto ins Spiel. Kurz das Konzept und Idee dahinter erläutern
		\end{itemize}
		\item Softwarekonzept
		\begin{itemize}
			\item Müsste nochmal genau abklären, was damit gemeint ist
		\end{itemize}
	\end{enumerate}
	\item Prototypische Implementierung 
	\item Fazit und Ausblick
\end{enumerate}

\chapter{Einleitung}

% % CHAPTER: Microservice - Observability - Governance
% \input{chapters/microservice_observability_gouvernance.tex}

% % CHAPTER: Konzept Service-Registry
% \input{chapters/microservice_dependency_graph.tex}

\chapter{Fazit und Ausblick}

%	Literaturverzeichnis
\clearpage
\ihead{}
\printbibliography[title=Literaturverzeichnis]
\cleardoublepage

% Der Anhang beginnt hier - jedes Kapitel wird alphabetisch aufgezählt. (Anhang A, B usw.)
% \appendix
% \ihead{\appendixname~\thechapter} % Neue Header-Definition


\end{document}
