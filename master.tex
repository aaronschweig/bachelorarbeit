\documentclass[
	12pt,
	BCOR=5mm,
	DIV=12,
	headinclude=on,
	footinclude=off,
	parskip=half,
	bibliography=totoc,
	listof=entryprefix,
	toc=listof,
	numbers=noenddot,
	plainfootsepline
]{scrreprt}

%	Konfigurationsdatei einziehen
\input{config}

\begin{document}

\TitelDerArbeit{Entwurf und Implementierung einer Referenzarchitektur zur automatischen Erzeugung digitaler Zwillinge}
\AutorDerArbeit{Aaron Schweig}
\Firma{Hays AG}
\Kurs{WWI18SEC}

\input{titlepage}


% Ehrenwörtliche Erklärung ewerkl.tex einziehen
\input{ewerkl.tex}

\pagenumbering{roman} % Römische Seitennummerierung
\normalfont

%	Kurzfassung
\chapter*{Kurzfassung}
\begingroup
\begin{table}[h!]
\setlength\tabcolsep{0pt}
\begin{tabular}{p{3.7cm}p{11.7cm}}
Titel: & \DerTitelDerArbeit \\
Verfasser/in: & \DerAutorDerArbeit \\
Kurs: & \DieKursbezeichnung \\
Ausbildungsstätte: & \DerNameDerFirma\\
\end{tabular}
\end{table}
\endgroup

In folgender Arbeit wird die automatische Erzeugung digitaler Zwillinge thematisiert. Anhand der Defintion eines digitalen Zwillings werden Anforderungen abgeleitet, die ein System mit sich bringen muss, um digitale Zwillinge abzubilden. Dafür werden nachdem Charakteristika digitaler Zwillinge herausgearbeitet wurden, Probleme definiert, die einer automatischen Erstellung im Wege stehen. Zunächst werden Kernkomponentenin Form einer Device Registry und eines Digital Twin Providers identifiziert. Anschließend wird ein Lösungsvorschlag auf Basis der Herausforderungen erstellt, der in der Einführung einer weiteren Komponente resultiert. In einer ausführlichen Analyse werden die identifizierten Probleme bearbeitet, sodass klare funktionelle Anforderungen an die einzelnen Komponenten gestellt werden können. Nachdem eine Analyse der einzelnen Komponenten durchgeführt wurde, wird anhand einer prototypischen Umsetzung die Validität des entwickelten Konzepts überprüft.



%	Inhaltsverzeichnis
\tableofcontents

%	Abbildungsverzeichnis
\listoffigures

%	Tabellenverzeichnis
% \listoftables

%	Listingverzeichnis
% \lstlistoflistings

% 	Abkürzungsverzeichnis (siehe Datei acronyms.tex!)
\clearpage
\chapter*{Abkürzungsverzeichnis}	
\addcontentsline{toc}{chapter}{Abkürzungsverzeichnis}


\begin{acronym}
	\acro{DHBW}{Duale Hochschule Baden-Württemberg}
	\acro{DT}{Digitaler Zwilling}
	\acro{PHM}{Prognostics \& Health Management}
\end{acronym}
\ohead{Acronyms} % Neue Header-Definition

%--------------------------------
% Start des Textteils der Arbeit
%--------------------------------
\clearpage
\ihead{\chaptername~\thechapter}
\ohead{\headmark}
\pagenumbering{arabic}

\chapter{Einleitung}

\begin{enumerate}
	\item Einleitung
	\item Grundlagen zu Digital Twins
	\begin{enumerate}
		\item Was sind DTs
		\begin{itemize}
			\item Unterscheidung verschiedener Arten und Abgrenzung zum CBS
		\end{itemize}
	\end{enumerate}
	\item Problemstellung und Lösungsansatz
	\begin{enumerate}
		\item Automatische Erzeugung digitaler Zwillinge
		\item Ist-Zustand in der aktuellen Umgebung
		\item Soll-Zustand
		\item Delta zwischen Ist und Soll
	\end{enumerate}
	\item Prototypische Implementierung 
	\item Fazit und Ausblick
	\begin{itemize}
		\item Mögliche Integration in die Eclipse Landschaft
		\item 
		\item ...
	\end{itemize}
\end{enumerate}

% % CHAPTER: Grundlagen DT's
\chapter{Grundlagen zu Digital Twins}

In folgendem Kapitel wird betrachtet, wie ein \ac{DT} definiert ist und welche Eigenschaften daraus ableitbar sind. Zudem wird versucht eine Unterscheidung und Abgrenzung zu einem Cyberphysischen-System vorzunehmen. Des weiteren werden verschiedene Anwendungsbereich innerhalb der Wirtschaft erläutert, welche alle von einem Einsatz digitaler Zwillinge profitieren würden. Dafür werden einige Beispiele angeführt.

\section{Was sind digitale Zwillinge?}

Innerhalb der Literatur lassen sich verscheidenste Ansätze erkennen, einen \ac{DT} zu beschreiben. Daraus leiten sich auch unterschiedliche Definitionen für einen \ac{DT} ab.

Laut \citeauthor{fuller2020digital} wurde der Begriff \enquote{Digitaler Zwilling} das erste mal von \citeauthor{grieves2014digital} im Jahre 2003 eingeführt, um später einem entsprechenden Whitepaper \citetitle{grieves2014digital} aus dem Jahr \citeyear{grieves2014digital} festgelegt zu werden.\autocite{fuller2020digital}


\chapter{Problemstellung und Lösungsansatz}

Im folgenden Kapitel soll das Problem, welches bei der automatischen Erzeugung digitaler Zwillinge besteht, beschrieben werden. Dafür wird zuerst ein Ablauf dargestellt anhanddessen ein \ac{DT} erstellt wird. Dabei wird dann auf den aktuellen Ist-Zustand eingegangen. Es wird außerdem ien Soll-Zustand definiert, welcher erreicht werden soll, sodass spätere Konzepte daran gemessen werden können. Am Ende dieses Kapitels wird evaluiert, wo die Diskrepanz zwischen dem aktuellen Ist-Zustand und dem Soll-Zustand liegt.

\section{Die automatische Erzeugung digitaler Zwillinge}

Das automatische Erzeugen digitaler Zwillinge stellt nur ein Teilziel in einem größeren Prozess dar, dessen Ziel es ist, mittels eines digitalen Zwilling eine dauerhaft ansprechbare Schnittstelle zu einen Endgerät zu haben. Bei dem Teilprozess der Erzeugung digitaler Zwillinge spielen zwei Komponenten eine wesentliche Rolle. In Abbildung \vref{fig:high_level} sind die Kernkomponenten sowie deren Zusammenhang aufgezeigt. Im speziellen spielen dabei die Device Registry und der Digital Twin Provider eine Rolle.

Da es sich bei diesen beiden Bestandteilen um voneinander unabhänige Anwendungen handelt, muss eine dauerhafte Verbindung sichergestellt werden. Außerdem müssen Geräte, die innerhalb der Device Registry liegen auch über den Digital Twin Provider bereitgestellt werden können. Für beide Komponenten gibt es spezifische Heruasforderungen, welche im folgenden näher betrachtet werden sollen.

\subsection{Device Registry}

Kernaufgabe der Device Registry ist es, wie in \vref{def:device_registry} noch einmal genauer beleuchtet, alle relevanten Endgeräte verwalten zu können. Um dieses Ziel erreichen zu können muss sichergestellt werden, dass die Vielzahl der im \ac{IoT} üblichen Kommunikationsprotokolle unterstützt wird. Würden nicht alle wichtigen Protokolle unterstützt werden, so wäre die Einsatzfähigkeit der Anwendung in Frage zu stellen. Zu den wichtigsten Protokollen gehören unter anderem:
\begin{itemize}
    \item MQTT
    \item AMQP
    \item HTTP/WS, etc.
\end{itemize}

\begin{figure}
    \centering
    \begin{tikzpicture}[node distance=4cm]
        \node (device) [io] {Endgerät};
        \node (dr) [process, right of=device, xshift=6cm] {Device Registry};
        \node (dts) [process, below of=dr] {Digital Twin Provider};
        \node (app) [startstop, below of=device] {Business Application};

        \draw [arrow] (device) -- node[anchor=south, align=left, text width=4cm] {sends telemetry data and authenticates} (dr);
        \draw [arrow] (dr) -- node[anchor=west, align=left, text width=2.5cm]{establishes connection and forwards telemetry data} (dts);
        \draw [arrow] (app) -- node[anchor=south, align=left, text width=4cm]{consumes information about digital twins} (dts);
        \draw [thick, dashed] (app) -- (device);
    \end{tikzpicture}
    \caption{Grobübersicht über die vorgeschlagene Architektur und deren Zusammenspiel\\ Quelle: Eigene Darstellung}
    \label{fig:high_level}
\end{figure}

Des weiteren gilt es die Herausforderung der Authentifizierung eines Endgerätes über die Device Registry zu lösen. Vor allem das Problem der Zugangsdaten muss gelöst werden, sodass ein problemloser Einsatz und Austausch von Endgeräten möglich ist, ohne administrativen Aufwand betrieben zu müssen. Dies hat auch Auswirkungen auf die wirtschaftlichen Auswirkungen, da ein geringer Wartungsaufwand auch geringere Kosten mit sich bringt. Ziel der in der Architektur verwendeten Device Registry sollte es dementsprechend sein, eine einfache, wartungsarme Authentifizierungsmethode bereitzustellen.

Außerdem muss eine Device Registry generisch einsetzbar sein. Die Praktikabiliät einer Lösung, welche nur Geräte eines Anwenders unterstützt und keine Möglichkeit bereitstellt, eine Sichtbarkeitseinstellung oder Authentisierung gewisser Geräte für bestimmte Anwender vorzunehmen würde den Einsatz einschränken. Auch hier würde eine Realisierung von mehr Anwendungsfällen zu höheren Kosten führen, da eventuell mehrere Serverressourcen bereitgestellt werden müssen.

\subsection*{Anforderungen an eine Device Registry}

Anhand obiger Analyse lassen sich mehrere Anforderungen an eine Device Registry ableiten:

\begin{enumerate}
    \item Bereitstellung einer einfachen Authentifizierungsmethode zwischen Endgerät und Device Registry
    \item Unterstützung möglichst vieler Kommunikationsprotokoll mit einer Möglichkeit zur Erweiterung um zusätzliche Protokolle
    \item Möglichkeit eines Einsatzes für mehrere Anwender innerhalb einer Device Registry. Eine Sichtbarkeitseinstellung muss möglich sein.
    \item Verarbeitung und Bereitstellung der Telemetriedaten der Endgeräte.
\end{enumerate}

Die Einführung einer Device Registry mit obigen Kriterien führt auch dazu, die Modularität zu erhöhen, da keine Funktionalität einer speziellen Device Registry vorausgesetzt wird. So kann auch im Falle eines Austauschs der Device Registry oder der Einführung einer Eigenentwicklung ohne Probleme migriert werden, sofern alle Kriterien erfüllt sind. Diese Modularität gilt es in allen Bereichen der Architektur zu erreichen.

\subsection{Digital Twin Provider}

Die zweite wichtige Komponente im Teilprozess der automatischen Erstellung digitaler Zwillinge stellt der \textbf{Digital Twin Provider} dar. Seine Rolle im Gesamtprozess wird noch einmal in Abbildung \vref{fig:high_level} aufgezeigt.

Die Kernaufgabe eines \ac{DT} Providers stellt das zur Verfügungstellen der \textit{eigentlichen} digitalen Zwillinge dar. Das bedeutet ihm fällt die Verwaltung der Verbindung zwischen der in der Device Registry vorhandenen Geräte und dem \ac{DT} zu. Er muss sicherstellen, dass jeder digitaler Zwilling immer ansprechbar ist und verwertbare Informationen liefert. Zusätzlich fällt ihm die Aufgabe zu, sicherzustellen, dass \ac{DT}s eine Struktur haben, sodass eine Anfrage konsitente Ergebnisse liefert.

Zusätzlich ist es wichtig sogenannte \textit{virtuelle Eigenschaften} auf Basis vorhandener Daten erstellen zu können. Diese Eigenschaften können mithilfe von Funktionen berechnet werden. Daraus folgt eine Anforderung an den Digital Twin Provider: Dieser muss die Möglichkeit bereitstellen, Funktionen innerhalb einer abgesicherten Umgebung zur Berechnung anlegen zu können.

Gleichzeitig gilt eine Anforderung, welche bereits für die Device Registry von Relevanz war. Es muss die Möglichkeit geben, den \ac{DT} Provider durch mehr als einen Anwender nutzen zu können. Dafür müssen Mechanismen bereitgestellt werden, welche im Idealfall mit denen der Device Registry übereinstimmen.

Der Digital Twin Provider muss außerdem ein umfangreiches \enquote{Policy} Modul integrieren, um Zugriffsregeln bis auf Eigenschaftenebene eines \ac{DT}s bestimmen zu können. Dies ist von immenser Bedeutung, um ein durchgängiges Authentifizierungs- und Authorisierungskonzept aufrechtezuerhalten und externe Business Applications nach dem \enquote{least visibility} Prinzip bedienen zu können, auch um schädliche Zugriffe zu vermeiden.

Business Applikationen verwenden unterschiedlichste Kommunikationsprotokolle, um Daten zu erhalten. Um sicherzustellen, dass auch hier eine Vielzahl von Applikationen bedient werden kann, müssen wie in der Device Registry möglichst viele Protokolle unterstützt werden und nach Möglichkeit erweiterbar sein. Die bedeutensten Formate für Business Applikationen sind:

\begin{itemize}
    \item \texttt{HTTP/1.x} und \texttt{HTTP/2.0}\footnotetext{HTTP 2 ermöglicht neben dem parallelen Übertragen von Daten auch das Streamen von Datenpaketen. HTTP 1 dagegen setzt nur ein klassisches Request-Response Pattern um.}
    \item Websockets
\end{itemize}

Diese Protokolle ermöglichen den einfachen Transfer von Telemetriedaten, sowohl auf Anfrage mittels \texttt{HTTP/1.x} oder auch kontinuierlich via Websockets oder \texttt{HTTP/2.0}.

Wichtig ist ebenfalls, das erzwingen einer konsitenten Struktur der digitalen Zwillinge. Dabei gilt es sicherzustellen, dass sowohl eine Struktur für die statischen Informationen (z.B. Modellnummer, Seriennummer, Standort, etc.) als auch die dynamischen Informationen (z.B. physische und virtuelle Sensoren) vorhanden ist. Dabei stellt es \textbf{nicht} die Aufgabe des \ac{DT} Providers dar, eine Verwaltungsstrutkur bereitzustellen, sondern nur auf Basis gegebener Strukturinformationen Validierungen durchzuführen und die Struktur zu erzwingen.

\subsection{Problemstellung}

Nun da die für das Problem relevanten Komponenten einmal näher beleuchtet wurden, sowie deren grundlegenden Fähigkeiten beschrieben wurden, kann näher auf die Problematik eingegangen werden, zu welcher diese Arbeit eine Löung bereitstellen soll. 

Die Device Registry stellt ein umfangreiches Verwaltungssystem für die physischen Geräte bereit und ist mit Schnittstellen ausgestattet, welche von anderen Applikationen genutzt werden können. Eine dieser Applikationen ist der Digital Twin Provider, der Informationen zu Geräten aus der Registry nutzen kann, um digitale Zwillinge zu konstruieren und bereitzustellen. Ebenfalls können die \ac{DT}s an dieser Stelle verwaltet werden. Allerdings können digitale Zwilling auch aufgrund der Anforderung einer festen Struktur \textbf{nicht} automatisch mit dem Senden der ersten Telemetriedaten nicht erstellt werden. Zusätzlich dazu stellt die Anforderung der genauen Definition einer Policy ein Problem bei der automatischen Erzeugung dar.

\begin{itemize}
    \item Herausforderung stellt das automatische Erstellen von DTs dar, dabei müssen verschiedene (Auth, Namespace, \textbf{Struktur}) Aspekte beachtet werden
    \item Frage: Wie kann ein automatisierter Prozess aussehen, der das Problem löst?
    \item Antwort: Das Bild und ein extra Connector
    \item Aktuell in Golang geschrieben, besser wäre allerdings eine Integration in das IoT Projekt von Eclipse
\end{itemize}

\begin{itemize}
    \item Trotz vieler verschiedener Anbieter von DTs werden diese nicht automatisch erzeugt, oder haben nicht den vollen funktionsumfang eines DT.
    \item Der Weg sollte direkt von einem selbstidentifizierenden Sensor stammen, der dann direkt in dem korrespondierenden Framework von DTs einen virtuellen Part bekommt.
\end{itemize}

\section{Ist Zustand}

\begin{itemize}
    \item DTs werden \enquote{manuell} erstellt.
    \item Sensoren haben ein fest definiertes Sensorenset; der DT ist darauf begrenzt $\rightarrow$ muss erweitert werden
    \item Mehr Automatisierung zur Erstellung von DTs unabhängig des verwendeten Schnittstellenformates. Viele verschiedene Austauschformate bringen eine erhebliche Last auf den DT-Provider
\end{itemize}
\chapter{Konzeptentwicklung}

Ziel des Konzeptes ist es eine möglichst automatisierte Erstellung digitaler Zwillinge ohne manuelle Interaktion bereitzustellen. Dies bringt verschiedene Vorteile mit sich, vor allem aber, können so dynamisch neue Zwillinge für neue Geräte generiert werden. Dabei ist es auch die Aufgabe bereits bestehende Geräte, sich im Falle einer Netzwerkpartition wieder mit ihrem entsprechenden Zwilling zu verbinden.\\
Dabei muss berücksichtigt werden, dass verschiedene Services bei dem Erstellen und dem Aufrechterhalten der Verbindung genutzt werden.

Eine wichtige Komponente stellt dabei die \textbf{Device Registry} dar. IBM definiert die Funktion einer Device Registry folgendermaßen:

\begin{definition}[Device Registry]
    \enquote{The Device Registry is the most important part of any IoT solution. With the registry, you can manage your device types and manage, maintain, and monitor the devices that you register.}\autocite{ibm_dr}\label{def:device_registry}
\end{definition}

Aus dieser Definition lassen sich mehrere wichtige Eigenschaften entnehmen. Zum einen stellt eine Device Registry einen zentralen Bestandteil dar. Dort liegt die Verwaltung der verscheidenen Geräte, welche alle entweder eigenständig oder miteinander kombiniert zur Erstellung digitaler Zwillinge genutzt werden können. Es geht auch hervor, dass die Art des Gerätes keine Rolle spielt - es können sogar verschiedene Typen verwaltet werden. Gleichzeitig dient es auch als zentrale Anlaufstelle, um die Authentizität von Geräten zu verifizieren. So muss eine Device Registry auch Mechanismen implementieren, die es ermöglichen tausende Geräte zu identifiezieren und zu verwalten.

Alle Bestandteile einer Device-Regisry lassen sich auf Abbildung \vref{fig:device_registry} erkennen. Für jeden Anwendungsfall eine eigene Device Registry aufzusetzen, Ressourcen für den Betrieb zu allokieren und zu warten gestaltet sich als unpraktisch. 
Deswegen sind die Registries verschiedener Anbieter \footnotetext{Google IoT Core, IBM, Eclipse Hono} mit einem Namespace System ausgestattet, sodass mehrere Nutzer abgetrennt voneinander innerhalb derselben Registry verwendet werden können.

\begin{figure}
    \centering
    \includegraphics[width=0.75\linewidth]{img/device-registry.png}
    \caption[Bestandteile einer Device Registry]{Beispielhafter Aufbau einer Device Registry am Beispiel von Eclipse Hono. Quelle: \url{https://www.eclipse.org/hono/docs/architecture/component-view/device-registry.png}}
    \label{fig:device_registry}
\end{figure}

Inhärent bedeutet das, dass das Authentifizierungs- und Authorisierungssystem dieses Scoping unterstützen muss. Somit benötigt die Umsetzung der Authentifizierung eine genauere Begutachtung.

\section{Authentifizierung und Authorisierung von IoT Geräten}
\label{sec:auth}

Die Authentifizerung von Geräten im IoT Bereich stellt eine Herausforderung dar. So ist es unpraktisch, für jedes Gerät eigene Zugangsdaten zu generieren und auf dem Gerät zu hinterlegen, damit es sich bei der Device Registry identifizieren kann. Auch birgt es ein Sicherheitsproblem, sollten sich Geräte eigenständig in der Device Registry registrieren können, ohne vorher Beweisen zu können, dass sie Zugriff auf einen bestimmten Namespace haben. \\
Es gilt also eine Alternative zu finden, welche es ermöglicht Geräte im großen Stil mit Zugangsdaten auszustatten, die allerdings nicht Geräteindividuell angepasst werden müssen.\\
Um das manuelle Hinterlegen geräteindividueller Zugangsdaten zu vermeiden müssen Alternativen untersucht werden, welche eine Vorauthentifizierung ermöglichen. 

Eigentliches Ziel der Authentifizierung ist der Nachweis einer behaupteten Eigenschaft einer Entität. Bricht man das Problem der Geräteauthentifizierung auf diesen Sachverhalt herunter, lässt sich erkennen, dass viele Konzepte der Krypthographie ähnliche oder gleiche Probleme bearbeiten. Ziel innerhalb der Kryptogrpahie ist unter anderen der Aufbau eines sicheren Kommunikationskanals zwischen verschiedenen Parteien. Umformuliert lässt sich das Problem folgendermassen darstellen:

\begin{problem}[Authentifizierung]
    Partei $A$ behauptet gegenüber Partei $B$, vertrauenswürdig zu sein.
\end{problem}

Dank dieser Transformation können nun Lösungen und Hilfsmittel aus der Krypthographie zur Bearbeitung des Authentifizierungsproblems herangezogen werden.

Grundsätzlich gilt es innerhalb der Kryptographie zwei Formen der Verschlüsselung zu unterscheiden:

\begin{enumerate}
    \item Die \textbf{symmetrische Verschlüsselung}, welche darauf beruht, dass zwei Schlüssel ausgetauscht werden, die im Anschluss genutzt werden können, um eine sichere Kommunikation aufrecht zu erhalten.
    \item  Die \textbf{asymmetrische Verschlüsselung}, die mit Hilfe eines öffentlichen und eines privaten Schlüssels eine sichere Kommunikation aufbaut und aufrecht erhält.
\end{enumerate}

Ein bekannter Nachteil symmetrischer Verschlüsselungsmethoden, ist der Schlüsselaustausch und das Sicherstellen einer sicheren Kommunikation zwischen einer wachsenden Anzahl an Teilnehmern. Somit wäre eine symmetrische Art der Verschlüsselung suboptimal für einen Anwendungsfall bei digitalen Zwillingen, da es sich bei den physischen Geräten um eine unbestimmte, sich stetig verändernde Anzahl handeln kann.

Alternativ bieten sich dann noch Möglichkeiten asymmetrischer Verschlüsselungsarten an. Ein Vorteil dieser Art ist es, dass ein Schlüsselaustausch sehr einfach erreicht werden kann, da der Aufbau der Verbindung auf Basis eines öffentlichen Schlüssels basiert. Nachteil dieser Methode ist es allerdings, dass der Verbindungsaufbau und die Geschwindigkeit der Verschlüsselung sehr langsam ist. Im speziellen wenn größere Datenmengen verschlüsselt werden müssen sind symmetrische Verschlüsselungen deutlich schneller. Außerdem stellen symmetrische Chiffren ein höheres Sicherheitsniveau zur Verfügung.

\subsection{Pre-Authentication via Zertifikat}
\label{sec:certificate}

Um soviele Vorteile wie möglich bei gleichzeitig so wenig Nachteilen wie nötig zu erhalten, ist es am sinnvollsten beide Ansätze miteinander zu kombinieren. Diese Idee wird bereits in der Internetkommunikation verwendet. Dabei wird mithilfe eines asymmetrischen Schlüsselpaars ein sicherer Kanal zur Kommunikation für einen symmetrischen Schlüsselaustausch und die anschließende Verschlüsselung aufgebaut. \\
Durch den Einsatz eines Zertifikates können die beschriebenen Vorteile erreicht werden.

\subsubsection*{Aufbau eines Zertifikates}

Durch \citeauthor{RFC3280} wurde ein Standard definiert, welcher zur Authentifizierung mit Zertifikaten genutzt werden soll. Dabei handelt es sich um den X.509 Standard. In \vref{fig:cert_structure} wird die grundlegende Struktur eines Zertifikates beschrieben. Die für den Anwendungsfall wichtigsten Informationen lassen sich dabei in der Struktur eines \texttt{TBSCertificate} finden. Dazu zählen die Informationen über den Herausgeber (Issuer) und den Zweck (Subject) des Zertifikats. \autocite{RFC3280}

\begin{definition}[Subject]
    Where it is non-empty, the subject field MUST contain an X.500 distinguished name (DN).  The DN MUST be unique for each subject entity certified by the one CA as defined by the issuer name field. \autocite[Kapitel~4.1.2.6]{RFC3280}
\end{definition}

\begin{verbatim}
Certificate  ::=  SEQUENCE  {
    tbsCertificate       TBSCertificate,
    signatureAlgorithm   AlgorithmIdentifier,
    signatureValue       BIT STRING  }
TBSCertificate  ::=  SEQUENCE  {
    version         [0]  EXPLICIT Version DEFAULT v1,
    serialNumber         CertificateSerialNumber,
    signature            AlgorithmIdentifier,
    issuer               Name,
    validity             Validity,
    subject              Name,
    subjectPublicKeyInfo SubjectPublicKeyInfo,
    issuerUniqueID  [1]  IMPLICIT UniqueIdentifier OPTIONAL,
                            -- If present, version MUST be v2 or v3
    subjectUniqueID [2]  IMPLICIT UniqueIdentifier OPTIONAL,
                            -- If present, version MUST be v2 or v3
    extensions      [3]  EXPLICIT Extensions OPTIONAL
                            -- If present, version MUST be v3
    }
Version  ::=  INTEGER  {  v1(0), v2(1), v3(2)  }
CertificateSerialNumber  ::=  INTEGER
Validity ::= SEQUENCE {
    notBefore      Time,
    notAfter       Time }
Time ::= CHOICE {
    utcTime        UTCTime,
    generalTime    GeneralizedTime }
UniqueIdentifier  ::=  BIT STRING
SubjectPublicKeyInfo  ::=  SEQUENCE  {
    algorithm            AlgorithmIdentifier,
    subjectPublicKey     BIT STRING  }
\end{verbatim}
\label{fig:cert_structure}
\begin{center}
    \small \fullcite{RFC3280}
\end{center}

Um massenhaft Geräte zu authentifizieren, kann während der Erstellung eines Namespaces/Tenants innerhalb der Device Registry ein Zertifikat inklusive korrespondierendem \textit{private Key} hinterlegt werden, dessen Zweck darin besteht, als Identifikator für Geräte zu dienen. Bei einem Verbindungsversuch kann dann seitens des Authentifizierungsservice überprüft werden, ob das mitgesendete Zertifikat für den angefragten Tenant gültig ist. Anschließend können dann über dieses \enquote{sicheren Kanal} gerätespezifische Zugangsdaten generiert oder ausgetauscht werden.

Mithilfe dieser Gerätezugangsdaten kann eine zukünftige Identifikation der Geräte durchgeführt werden.

\begin{figure}[ht]
    \centering
    \includegraphics[width=1.0\linewidth]{img/device_authentication.png}
    \caption[Ablauf der Authentifizierung via Zertifikat]{Ablauf der Authentifizerung eines Gerätes via Zertifikat.\\ Quelle: Eigene Darstellung}
    \label{fig:certificate}
\end{figure}

In Abbildung \vref{fig:certificate} ist zu erkennen, wie der Ablauf einer solchen Authentifizierung sein kann.

\subsection{Direkter Einsatz der Device Zugangsdaten}
\label{sec:credentials}

Werden die Zugangsdaten direkt auf dem Gerät hinterlegt, entsteht das oben beschriebene Problem. Das vorherige generieren dieser Zugansdaten bringt verschiedene Probleme mit sich, wenn die Geräte, die im späteren System eingesetzt werden, nicht final bekannt sind. Zudem muss außerdem noch das Problem gelöst werden, wie Zugangsdaten auf ein Gerät gelangen können.

Dieser Nachteil kann allerdings ignoriert werden, sollten es eigene Geräte sein, oder falls sich alle Geräte vor einem produktiven Einsatz in einer \enquote{Einrichtungsphase} befinden. Mithilfe dieser Methode könnte die Aufgabe der Zertifikaterstellung und -validierung von der Device Registry entfernt werden, was ein insgesamt klareres System erzeugt.

\subsection{Vergleich beider Authentifizierungsvarianten}

Um einen optimalen Prozess zur Authentifizierung von IoT Geräten zu gewährleisten kann eine Kombination der beiden Varianten genutzt werden. Dabei wird zuerst das Zertifikat verwendet, um sicherzustellen, dass das Gerät berechtigt ist, gerätespezifische Zugangsdaten anzufordern. Anschließend wird mithilfe des Zertifikates die zugehörigen zu einem bestimmten Namespace überprüft. Anschließend kann das Gerät, falls es noch nicht existiert, registriert werden. Danach werden die gerätespezifischen Zugangsdaten von de, Authentication Server generiert und an das Gerät zurückgegeben. Die folgende Kommunikation zwischen dem Gerät und den anderen Komponenten kann danach mithilfe dieser Informationen erfolgen.

Entsprechend des Anfangs vorgestellten Modells, stellt die Authentifizierung via Zertifikat (asymmetrisch) einen sicheren Kanal zum Austausch der Zugansdaten (symmetrisch) bereit. Der gesamte Authentifizerungsvorgang ist in Abbildung \vref{fig:dev_reg_proc} beschrieben. Außerdem ermöglicht dieses Modell auch die Authentifizierung ohne Zertifikat. Das hat allerdings zur Folge, dass die Zugansinformationen bereits auf dem Gerät hinterlegt sein müssen, was inhärent bedeutet, dass das Gerät bereits in der Device Registry registriert sein muss.

\begin{figure}
    \centering
    \includegraphics[width=0.8\linewidth]{img/device_registration.png}
    \caption[Device Registration Process]{Device Registration Process.\\Quelle: Eigene Darstellung}
    \label{fig:dev_reg_proc}
\end{figure}

\section{Sicherstellen der Device Connectivity in der Device Registry}

Der Device Registry fällt ebenfalls die Aufgabe zu, sicherzustellen, dass eine Vielzahl verschiedener Kommunikationsprotokolle unterstützt wird, damit ein Verbindungsaufbau mit einer Vielzahl von Geräten möglich ist. Im folgenden werden verschiedene Kommunikationsprotokolle vorgestellt, sowie kurz deren jeweilige Vor- und Nachteile beleuchtet. Im nächsten Schritt wird die Wahl eines asynchronen, internen Kommunikationsmittels beschrieben.

\subsection{MQTT}
Bei dem \ac{MQTT} handelt es sich um ein sehr populäres Kommunikationsprotokoll, das in den letzten Jahren immer mehr an Popularität gewonnen hat. Das \ac{MQTT} Protokoll definiert folgende Bestandteile:

\begin{definition}[MQTT]
    MQTT\autocite{standard2014mqtt} is a Client Server publish/subscribe messaging transport protocol. [...]These characteristics make it ideal for use in many situations, including constrained environments such as for communication in Machine to Machine (M2M) and Internet of Things (IoT) [...].

    The protocol runs over TCP/IP [...]. Its features include:

    \begin{itemize}
        \item Use of the publish/subscribe message pattern which provides one-to-many message distribution and decoupling of applications.
        \item A messaging transport that is agnostic to the content of the payload.
        \item A small transport overhead and protocol exchanges minimized to reduce network traffic.
        \item A mechanism to notify interested parties when an abnormal disconnection occurs.
    \end{itemize}
\end{definition}

Bei der Nutzung des MQTT-Protokolls werden zwei Bestandteile benötigt. Zum einen der \textbf{MQTT-Client}, welcher auf den Endgeräten genutzt wird und eine Kommunikation mit dem \textbf{MQTT-Broker} aufbaut.\\ Kernaufgabe des Brokers ist es dabei, die verschiedenen Topics inklusive den zugehörigen Subscribern und Zustellungsvorgaben zu verwalten. Um eine möglichst hohe Ausfallsicherheit zu gewährleisten, können mehrere Broker bereitgestellt werden. Mithilfe entsprechender Konfiguration ist es möglich, im Falle eines Ausfalls auf andere Broker auszuweichen, ohne einen Verbindungs- oder Kommunikationsverlaust zu erleiden.

Wird das \ac{MQTT} Protokoll innerhalb einer Device Registry genutzt wird nicht der gesamte Funktionsumfang eines Brokers benötigt. Wichtig ist dabei nur, dass Nachrichten, die dem \ac{MQTT} Standard folgen, von der Registry korrekt interpretiert werden können.\\
Dabei gilt es auch zu unterscheiden, ob es sich um unidirektionale oder bidirektionale Kommunikation handelt. 

\subsection*{Unidirektionale Kommunikation}
Ziel dieser Form der Kommunikation innerhalb dieses spezifischen Anwendungsfalls ist das Senden der Telemetrieinformationen von dem Gerät zu der Device-Regitry, welche im diesem Fall die Rolle des Brokers einnimmt.

Dabei werden hauptsächlich \textit{Telemetrieinformationen} des Gerätes an die Device Registry weitergegeben. Der Rückkanal wird dabei nur von den durch das Protokoll beschriebenen Acknowledgements genutzt.

\begin{figure}[h]
    \centering
    \begin{tikzpicture}[node distance=2cm]
        \node (dev1) [io] {Endgerät};
        \node (dev2) [process, right of=dev1, xshift=6.5cm] {MQTT-Adapter der Device Registry};
        \draw [arrow] (dev1) -- node[anchor=south] {Telemetry} (dev2);
        \draw [arrow] (dev2) -- node[anchor=north] {Ack} (dev1);
        \draw [thick, dashed] (dev2) -- (13,0);
    \end{tikzpicture}
    \caption[Unidirektionale Kommunikation]{Unidirektionale Kommunikation zwischen Endgerät und Device Registry.\\Quelle: Eigene Darstellung}
\end{figure}

\subsection*{Bidirektionale Kommunikation}
Im Gegensatz zu der unidirektionalen Kommunikation steht bei der bidirektionalen Kommunikation die Informationsübermittlung zwischen beiden Komponenten im Vordergrund. So werden nicht nur Telemetrieinformationen des Endgerätes an die Device-Registry übermittelt, sondern auch Befehle o.ä. an das Endgerät.

\begin{figure}[h]
    \centering
    \begin{tikzpicture}[node distance=2cm]
        \node (dev1) [io] {Endgerät};
        \node (dev2) [process, right of=dev1, xshift=6.5cm] {MQTT-Adapter der Device Registry};
        \draw [arrow] (dev1) -- node[anchor=south] {Telemetry} (dev2);
        \draw [arrow] (dev2) -- node[anchor=north] {Kommandos, etc.} (dev1);
        \draw [thick, dashed] (dev2) -- (13,0);
    \end{tikzpicture}
    \caption[Bidirektionale Kommunikation]{Bidirektionale Kommunikation zwischen Endgerät und Device Registry.\\Quelle: Eigene Darstellung}
\end{figure}

Eine Besonderheit der bidirektionalen Kommunikation mittels \ac{MQTT} stellt das fehlende Request-Response Pattern dar. Bei anderen Protokollen (wie z.B. HTTP) ist in das Protokoll ein Mechanismus eingebaut, um auf die Anfrage eines Clients eine Antwort an nur diesen Client zurückzugeben. Dieses Konzept wurde innerhalb der MQTT Spezifikation nicht umgesetzt. Stattdessen kann dort mittels klar definierten Topic Strukturen ein Request-Reponse Pattern integriert werden.
\pagebreak
\begin{example}[Request-Response innerhalb von MQTT] Eine examplarische Methode zur Integration eines Request-Response Pattern kann folgender Abbildung entnommen werden. Dabei ist zu erkennen, dass das Zeil mithilfe einer bestimmten Topicstruktur erreicht wird. Diese Topicstrukturen können innerhalb von MQTT nicht erzwungen werden, sodass nicht sichergestellt ist, dass dieses Konzept zuverlässig funktionert.

    \begin{figure}[h]
        \centering
        \begin{tikzpicture}[node distance=2cm]
            \node (dev1) [process] {Endgerät};
            \node (dr1) [process, right of=dev1, xshift=8.5cm] {MQTT-Broker};
            \node (dev2) [process, below of=dev1] {Endgerät};
            \node (dr2) [process, right of=dev2,below of=dr1, xshift=-2cm] {MQTT-Broker};

            \draw [arrow] (dev1) -- node[anchor=south] {request/<unique identifier>} (dr1);
            \draw [thick] (dev2) -- node[anchor=south] {subscribe response/<unique identifier>} (dr2);
        \end{tikzpicture}
        \caption[Request-Response Pattern innerhalb von MQTT]{Beispielhafte Umsetzung eines Request-Response Patterns innerhalb von MQTT.\\Quelle: Eigene Darstellung}
    \end{figure}
\end{example}

Zur Nutzung des \ac{MQTT} Protokolls muss eine Device Registry eine Möglichkeit bereitstellen, via MQTT mit Endgeräten zu kommunizieren. Um dieses Ziel zu erreichen kann ein Broker verwendet werden. Ein sehr populärer Vertreter eines Brokers stellt das Eclipse Mosquitto Projekt dar.\\
Gleichzeitig muss eine authentifizierte Kommunikation gewährleistet werden können. Wie in \vref{sec:auth} beschrieben, wird vorgelagert mittels eines Zertifikats sichergestellt, dass das Endgerät eindeutig identifizierbare Zugangsdaten besitzt. Diese Zugangsdaten können verwendet werden, um mithilfe eines eindeutigen Identifikators des Namespaces die \texttt{username} und \texttt{password} Informationen während der Kommunikation zu nutzen.

\subsection{AMQP}

\begin{definition}
    \enquote{AMQP is a lightweight M2M protocol, which was developed by John O’Hara at JPMorgan Chase in London, UK in 2003. It is a corporate messaging protocol designed for reliability, security, provisioning and interoperability. AMQP supports both request/response and publish/subscribe architecture.}\autocite{naik2017choice}
\end{definition}

Ähnlich wie bei \ac{MQTT} gibt es innerhalb von \ac{AMQP} mehrere Akteure. Es gibt einen sogenannten \textit{Produzenten} der Nachricht, welche die zu übertragenden Informationen spezifiziert. Die Nachricht stellt dabei das Kernelement der gesamten Kommunikation dar. Diese Nachricht wird dann in einem Message Broker verarbeitet. Die Aufgaben des Message Brokers ist dabei äquivalent zu dem \ac{MQTT} Protokoll. Ein \textit{Konsument} kann sein Interesse an bestimmten Nachrichten bekunden, indem er auf sogenannte \textit{Exchanges} subscribed. Der Konsument hat somit die Möglichkeit Nachrichten, die innerhalb dieses Themas publiziert wurden anzufordern und zu verarbeiten.

\ac{AMQP} unterscheidet sich allerdigns in der Funktionsweise des Brokers von \ac{MQTT}. Innerhalb des Brokers gibt es drei Kernbestandteile:
\begin{enumerate}
    \item Die \textbf{Exchange}:\\
    Die Exchange definiert die Art der Nachrichtenübermittlung, im speziellen, ob es sich dabei um einen \textbf{Direct Exchange}, oder einen \textbf{Fanout Exchange} handelt. Der Direct Exchange stellt Nachrichten an einen einzigen Empfänger durch, wohingegen der Fanout Exchange die Nachricht an alle interessierten Queues vervielfältigt. Es gibt also zwei verschiedene Arten der Kommunikation mit einem oder wenigen Empfängern. Im Gegesnatz zu den obigen Methoden steht der sog. \textbf{Topic Exchange}. Der Topic Exchange ermöglicht es Platzhalter zu verwenden, um so nicht auf alle spezifischen Routen subscriben zu müssen.
    \item Die \textbf{Routes}:\\
    Routen stellen die Verbindung zwischen der Exchange und er Queue dar. Sie leiten Nachrichten, welche in einer Exchange zur Verfügung gestellt werden, zu den interessierten Queues weiter. Entsprechend der gewählten Art der Exchange werden entsprechende Routing Regeln angelegt. Sie dienen daher als reine Nachrichtenübermittler
    \item Die \textbf{Queues}:\\
    Queues können entweder von dem Broker oder von einem Client mit einem Identifikator versehen werden. Queues stellen einen physischen Speicher dar, welcher als Speicher für die Nachrichten dient, die mittels einer Exchange publiziert werden. Clients interagieren direkt mit der Queue, um Nachrichten zu empfangen.
\end{enumerate}

In der Abbildung \vref{fig:message_flow_amqp} ist der Zusammenhang der verschiedenen Komponenten anhand der Verarbeitung einer Nachricht visualisiert.

\begin{figure}
    \centering
    \begin{tikzpicture}[node distance=4cm]
        \node (publisher) [io] {Publisher};
        \node (exchange) [process, right of=publisher] {Exchange};
        \node (route) [process, right of=exchange] {Route};
        \node (queue) [process, below of=route] {Queue};
        \node (client) [io, below of=publisher] {Client};
    
        \draw [arrow] (publisher) -- (exchange);
        \draw [arrow] (exchange) -- (route);
        \draw [arrow] (route) -- (queue);
        \draw [arrow] (queue) -- (client);
    \end{tikzpicture}
    \caption{Verarbeitung einer Nachricht über einen \ac{AMQP} Message Broker.\\Quelle: Eigene Darstellung}
    \label{fig:message_flow_amqp}
\end{figure}

Der Nachteil eines fehlenden Request-Response Patterns von \ac{MQTT} ist dank der \textbf{Direct Exchange} direkt in \ac{AMQP} gelöst. Dies ermöglicht zusätzlich zu allen Funktionen, welche von \ac{MQTT} angeboten werden, einen erweiterten Einsatzbereich.

\subsection{LoRaWAN}
% \subsection{Anforderungen an ein internes Kommunikationsmittel}

% \begin{itemize}
%     \item Anfragestruktur mit Pufferlösung ist wichtig, damit bei vielen Anfrage keine Verloren geht.
%     \item HTTP fällt weg, da nicht pufferbar und es iwann einen timeout gibt, wenn die anfrage nicht beantwortet wird. Websockets wären eine alternative, dabei würde aber die Last massiv auf dem/den Servern liegen, was zu Problemen führen kann. eleganter wäre eine Lösung mittels eines Message Brokers. Welche kommen in Frage?
%     \item MQTT sehr bekannt, oft eingesetzt, hat aber keinen Support for Req-Res pattern, was in einem Microserviceumfeld schwierigkeiten bereitet.
%     \item AMQP bietet sich an, da schnell (binary Protocol), unterstützt Req-Res Pattern und ist Brokerbasiert, sodass anfragen gepuffert werden und asynchron abgearbeitet werden können.
% \end{itemize}

\begin{enumerate}
    \item Architekturkonzept
    \begin{itemize}
        \item Unterscheidung zwischen einer Device-Registry und der Speicherung als DT
        \begin{itemize}
            \item Hier liegt der Übergang der Arbeit; Es müssen von Geräten in der Device Registy DTs gebildet werden, nach möglichkeit automatisiert, sodass dann Daten des Gerätes übertragen werden
        \end{itemize}
        \item Was muss eine DT-Architektur mit sich bringen, damit Definition erfüllt Ist?
        \begin{itemize}
            \item Hier ein wenig auf die Ditto Idee eingehen.
            \item Konzept hinter Connections erklären; Was sind Connections, Was bringt das an Vorteilen mit sich
            \item Kombination zwischen Device-Registry und DT (hier liegt das Kernproblem!)
        \end{itemize}
        \item Welche Daten sind vorhanden und wie kann sichergestellt werden, dass ein speziell Modell für DT's vorliegt? $\rightarrow$ Hier kommt dan Vorto ins Spiel. Kurz das Konzept und Idee dahinter erläutern
        \begin{itemize}
            \item Vorto als Strukturregistry.
            \item Wieso ist das so wichtig
            \item welche vorteile gibt es, ein solches ding zu haben
        \end{itemize}
    \end{itemize}
\end{enumerate}

\section{Wie kann das alles genutzt werden? -- ADAPTER}

\section{Digital Twin Provider}

In Folgendem Abschnitt soll der genaue Aufbau eines Digital Twin Providers beschrieben werden. Dabei wird im genaueren Betrachtet wie die Anforderungen aus \vref{sec:dtp} umgesetzt werden können.

\begin{figure}
    \centering
    \includegraphics[width=0.8\linewidth]{img/ditto_arch.png}
    \caption[Aufbau eines Digital Twin Providers]{Aufbau eines Digital Twin Providers am Beispiel von Eclipse Ditto.\\Quelle: \url{https://www.eclipse.org/ditto/images/pages/architecture/context-overview.png}}
    \label{fig:ditto_arch}
\end{figure}

\subsection*{Integration von Strukturinformationen}
Eine der wichtigsten Funktionen des Digital Twin Providers stellt die strukturierte Bereitstellung der gesammelten Telemetriedaten dar. Da digitale Zwillinge nicht zwingend dem zugrundeligenden physischen Sensor entsprechen, sondern auf beliebige Art und Weise erweitert werden können, sind Strukturinformationen von elementarer Bedeutung, um den Nutzen zu erhöhen. Die Integration dieser Informationen muss optional sein, da sonst Einstiegshürden geschaffen werden, welche die \enquote{time to market} einer zu entwickelnden Applikation negativ beeinflussen können. Daher bietet es sich an eine extra Komponente in der Gesamtarchitektur bereitzustellen, welche optional hinzugefügt und integriert werden kann.

\begin{figure}
    \centering
    \begin{tikzpicture}[node distance=4cm]
        \node (device) [io] {Endgerät};
        \node (dr) [process, right of=device, xshift=2cm] {Device Registry};
        \node (dts) [process, below of=dr] {Digital Twin Provider};
        \node (vorto) [process, right of=dts] {Strukturregistry};
        \node (connector) [process, fill=green!30, right of=dr, yshift=-1.75cm] {Connector};
        \node (app) [startstop, below of=device] {Business Application};
    
        \draw [arrow] (device) -- (dr);
        \draw [arrow] (dr) -- (dts);
        \draw [arrow] (app) -- (dts);
    
        \draw [thick, dashed, color=red] (connector) -- (dr);
        \draw [thick, dashed, color=red] (connector) -- (dts);
        \draw [dashed] (connector) -| (device);
        \draw [dashed, thick] (vorto) -- (dts);
    \end{tikzpicture}
    \caption[Integration der Strukturregistry in die Gesamtarchitektur]{Integration der Strukturregistry in die Gesamtarchitektur.\\Quelle: Eigene Darstellung}
\end{figure}

Das einfügen einer Strukturregistry ermöglicht es, die Komplexität verschiedener Austauschformate zu reduzieren. Mithilfe einer Mapping-Engine, können Regeln definiert werden, die ein Eingangsformat in ein einheitliches Format zur Nutzung innerhalb einer Applikation transfomieren. Dies hat den Vorteil, die Kopplung zwischen Endgerät, dem korrespondierenden \ac{DT} und der Applikation zu verringern. Die nötige Logik, um die verschiedenen Datenformate zu interpretieren wird von der Strukturregistry übernommen und muss nicht in jeder konsumierenden Applikation umgesetzt werden.

Des weiteren können so ohne Anpassungen in einer Applikation vornehmen zu müssen weitere Gerätetypen genutzt werden. Selbst wenn sich die komplette Sensorenlandschaft ändert, z.B. bei einem Anbieterwechsel, o.ä., kann mithilfe eines Mappings innerhalb der Strukturregistry sichergestellt werden, dass das neue Datenformat in das einheitliche, von allen Applikationen verstandene Format transformiert wird.

Zusammenfassend lassen sich folgende Vorteile mit der Nutzung einer Strukturregistry erreichen:

\begin{enumerate}
    \item Die time to market wird reduziert, da Entwicklungsaufwände sinken und durch klare Datenformate parallelisiert werden können.
    \item Wartungsarbeiten oder spätere Anpassungen auf neue Anforderungen sind einfacher umsetzbar, da einheitliche Datenformate vorhanden sind.
\end{enumerate}

Sollte es nicht möglich sein, eine unabhängige Strukturregistry bereitzustellen, können diese Vorteile trotzdem erreicht werden. Dafür muss der Digital Twin Provider eine Mapping Engine anbieten, die es ermöglicht Telemetriedaten, die von der Device Registry empfangen werden, zu transformieren. So kann auch ohne den Einsatz einer dedizierten Strukturregistry der Vorteil einer einheitlichen Datenstruktur erreicht werden.\\
Ein Nachteil dieser leichtgewichten Variante ist jedoch, dass die Kopplung zwischen Digital Twin Provider und Device Registry noch einmal erhöht wird, was die Modularität der Architektur negativ beeinflusst.

\subsection{Verwaltung von digitalen Zwillingen}

Ein weiterer wichtiger Aspekt ist die Verwaltung digitaler Zwillinge. Der Digital Twin Provider erhält die Telemetriedaten über die Device Registry. Es gilt aber zusätzlich statische Informationen über den \ac{DT} zu verwalten. Dafür müssen entsprechende API Schnittstellen bereitgestellt werden.

Auch wird eine erweiterte Suchfunktionalität benötigt, um spezielle Gruppen digitaler Zwillinge einfach identifizieren zu können. Dies ermöglicht applikationspezifische Logik mithilfe von Suchanfragen umzusetzen. Ein exemplarische Aufbau eines digitalen Zwillings innerhalb eines Digital Twin Providers kann folgendermaßen aussehen:

\begin{verbatim}
{
  "thingId": "the.namespace:theId",
  "policyId": "the.namespace:thePolicyId",
  "definition": "org.eclipse.ditto:HeatingDevice:2.1.0",
  "attributes": {
      "someAttr": 32,
      "manufacturer": "ACME corp"
  },
  "features": {
      "heating-no1": {
          "properties": {
              "connected": true,
              "complexProperty": {
                  "street": "my street",
                  "house no": 42
              }
          },
          "desiredProperties": {
              "connected": false
          }
      },
      "switchable": {
          "definition": [ "org.eclipse.ditto:Switcher:1.0.0" ],
          "properties": {
              "on": true,
              "lastToggled": "2020-11-15T18:21Z"
          }
      }
  }
}
\end{verbatim}

Innerhalb der Struktur lassen sich einige umgesetzte Anforderungen erkennen. Eine Anforderung, welche bereits in der Device Registry eine große Rolle spielte, war eine Umsetzung eines Namespacing Konzepts. Dieses Konzept wird in dem Digital Twin provider mithilfe fester Präfixe erreicht. Exemplarisch dafür kann die \texttt{thingId} betrachtet werden. Der Namespace wird dabei mittels eines Doppelpunktes von dem eigentlichen \enquote{Namen} des \ac{DT}s abgetrennt. \\
Dieses Konzept kann auch bei anderen Ressourcen des Digital Twin Providers genutzt werden. Somit wird erreicht, dass der Provider von mehreren Anwendern gleichzeitig genutzt werden kann.

Des weiteren lässt sich die Persistenz der dynamischen und statischen Daten erkennen. Innerhalb der \texttt{attributes} sind alle statischen Informationen über den digitalen Zwilling gespeichert. Hierbei ist prinzipiell keine feste Struktur vorgegeben und es können beliebe Werte gespeichert werden. Diese Werte können zur weiteren Identifikation genutzt werden. Gleichzeitig können diese Werte bei der Suche und Gruppierung von digitalen Zwillingen eingesetzt werden.

\subsection{Integration von Policies zur Zugriffsbeschränkung}

Wie bereits in der Datenstruktur zu erkennen ist, muss dem digitalen Zwilling eine Policy zugewiesen werden. Die Aufgabe einer Policy besteht darin, Zugriffe auf Attribute oder den gesamten digitalen Zwilling zu steuern. So kann sichergestellt werden, dass Informationen nur berechtigten Applikationen zur Verfügung gestellt werden. Die Struktur einer Policy kann folgendermaßen aussehen:

\begin{verbatim}
{
  "policyId": "my.namespace:policy-a",
  "entries": {
    "observer": {
      "subjects": {
        "nginx:observer-client": {
          "type": "technical client"
        },
        "nginx:some-users": {
          "type": "a group of users"
        }
      },
      "resources": {
        "thing:/features/featureX": {
          "grant": ["READ"],
          "revoke": []
        },
        "thing:/features/featureY": {
          "grant": ["READ"],
          "revoke": []
        }
      }
    }
  }
}
\end{verbatim}

Die Umsetzung eines policy-basierten Zugriffssystems muss eine möglichst hohe Dynamik haben, damit die dynamische Struktur der Daten eines digitalen Zwillings unterstützt werden kann. Es lässt sich ausserdem erkenne, dass auch hier bei der \texttt{policyId} wieder dasselbe Konzept zum Namespacing genutzt wird, wie auch bei der Speicherung des \ac{DT}s an sich. Es können verschiedene \enquote{Rollen} innerhalb der Policy definert werden, welche unterschiedliche Nutzergruppen repräsentieren können. Grundsätzlich folgt die Implementierung aber dem Konzept einer \ac{ACL}. Es werden also \textbf{explizit} alle Nutzer benannt, die Zugriff auf eine Ressource haben. Diese werden innerhalb des \texttt{subject} Keys definiert. Die \texttt{resources} beschreiben ebenfalls \textbf{explizit} welche Operationen genehmigt oder verweigert werden. Die Ressourcen können mittles einer dynamischen Syntax angepasst werden.

\subsection{Verbindung zwischen Digital Twin Provider und Device Registry}

Als letzte wichtige Komponente wird noch einmal der zuständige Service zur Datenübermittlung zwischen Device Registry und dem Digital Twin Provider betrachtet. Eclipse Ditto beschreibt den Nutzen und die Funktionsweise dieses Services folgendermaßen:

\enquote{The “connectivity” service enables Ditto to establish and manage client-side connections to external service endpoints. You can communicate with your connected things/twins over those connections via Ditto Protocol messages. The connectivity service supports various transport protocols, which are bound to the Ditto Protocol via specific Protocol Bindings.}\autocite{ditto}

Die Unterstützung mehrerer Kommunikationsprotokolle ermöglicht die Nutzung des Providers mit verschiedenen Device Registries. Wichitg bei der Umsetzung eins Connectivity Service ist die verschiedenen Authentifizierungsarten zu beachten, die eventuell von der Device Registry unerstützt werden. Zudem muss ein gemeinsames Protokoll zur Authentifizierung ausgewählt werden, um eine sichere Kommunikation zwischen den beiden Parteien sicherzustellen.

Im folgenden Teil wird die Realisierung der Architektur mithilfe verschiedener Eclipse Projekte beschrieben. Dabei werden unter anderem Eclipse Hono(Device Registry) und Eclipse Ditto(Digital Twin Provider) verwendet. Die Kommunikation zwischen diesen beiden Komponenten wird mithifle von \ac{AMQP} sichergestellt. So können Telemetriedaten, welche in Eclipse Hono erhalten werden über die PubSub Funktion von AMQP direkt an Eclipse Hono weitergeleitet werden. Ditto kann dann intern sicherstellen, dass der digitale Zwilling den aktuellen Status wiederspiegelt.

\clearpage
\section{Softwarekonzept}

Wie bereits beschrieben soll nun das theoretische Konzept der Architektur mithilfe von konkreten Softwareimplementierungen angereichert werden, um ein funktionsfähiges System abzubilden. Um eine möglichst hohe Barrierefreiheit bei der Umsetzung der in dieser Arbeit beschriebenen Architektur zu gewährleisten wird bei der Wahl der Software auf eine Open Source Verfügbarkeit gesetzt. Dies ermöglicht die freie Nutzung für jeden Anwender. Besonders wird dabei 

\subsection{Eclipse Hono}

Als Vertreter für die Device Registry kann Eclipse Hono\footnote{\url{https://www.eclipse.org/hono/}} verwendet werden. Hono ist ein geeigneter Vertreter, da es die Anforderungen bzgl. Authentifizerung und unterstützter Protokolle erfüllt. Die interne Aufbau von Hono als Device Registry wurde bereits in Abbildung \vref{fig:device_registry} näher beleuchtet. Hono unterstützt Zertifikate als Möglichkeit zur Authentifizierung.

Um Zertifikate nutzen zu können, müssen allerdings bestimte Vorraussetzungen erfüllt sein. Da ein Zertifikat auch als Informationsquelle zur Identifizierung des Koresspondieren Tenants dient, muss bei der Erstellung des Tenants angegeben werden, welche Zertifkate genutzt damit genutzt werden können. Dies erfordert das manuelle Erstellen eines Zertifkates. Sobald dieses allerdings verfügbar ist, kann der Tenant erfolgreich erstellt werden, und entsprechende Ableger des Zertifikates können auf den Endgeräten hinterlegt werden und zur Authentifizierung genutzt werden. Damit kann der Ablauf aus Abbildung \vref{fig:dev_reg_proc} mithilfe von Eclipse Hono realisiert werden.

Wie bereits erwähnt unterstützt Eclipse Hono auch eine Vielzahl an Kommunikationsprotokollen, sodass eine große Anzahl an Geräten unterstützt werden kann. Zu den Protokollen zählen unter anderem:

\begin{itemize}
    \item MQTT
    \item HTTP
    \item AMQP
    \item CoAP
    \item Sigfox
    \item LoRaWAN
\end{itemize}

Die Unterstützt dieser Protokolle ist mithilfe von Adaptern realisiert. Das bedeutet, es können auch, sollten eigene oder neue Protokolle unterstützt werden, neue Adapter implementiert werden. Dies ermöglicht eine hohe Anpassbarkeit, die je nach Anwendungsfall gewährleistet ist.

\subsection{Eclipse Ditto}


\begin{itemize}
    \item Hono als DR
    \item Ditto als DT Provider
    \item Vorto als Strukturrepository
    \item \enquote{Am Beispiel eines Sensors mit LoRaWAN und TTN}
    \item Beschreiben der Schritte innerhalb der Umsetzung (Go-App)
    \item 
\end{itemize}

\large{Deployment}
\normalsize
\begin{itemize}
    \item Deplomyent via Container (Docker)
    \item verschiedene Methoden der Containerorchestrierung (docker-compose, kubernetes)
    \item Hinzufügen von Hard- und Softwaremonitoring
\end{itemize}


\chapter{Fazit und Ausblick}

In dieser Arbeit wurde ein Konzept vorgestellt, welches das automatische Erstellen digitaler Zwillinge ermöglicht. Dabei lag der Fokus vor allem darauf, das Zusammenspiel der einzelnen Komponenten zu untersuchen und Problemstellen zu identifizieren. Anschließend konnten aber auch Synergien festgestellt und genutzt werden. Gerade im Berech der Authentiizierung mussten einige Herausforderungen bewältigt werden. Unter anderem wurde dabei eine Methode vorgestellt, wie sich Geräte mithilfe eines Zertifikates ausweisen können, um anschließend mithilfe vor geräteindividuellen Zugansdaten operien zu können. Diese Methode unterstützt die dynamische Umgebung, und die sich ständig ändernde Anzahl an genutzten Geräten, ohne dass ein komplexer Registrierungsprozess nötig ist. Danach konnten die verschiedenen Komponenten zu einer Architektur zusammengeführt werden und mithilfe eines neuen Service erweitert werden, welcher die Aufgabe der Erstellung der digitalen Zwillinge übernimmt. 

Gerade im Bereich des zusätzlich entwickelten Services können im Rahmen weiterer Arbeiten einige Verbesserungen vorgenommen werden. In der aktuellen Implementierung wird nur eine bestimmte Kombination von Device Registry und Digital Twin Provider ohne die Verwendung einer Sturkturregistry ermöglicht. In einer Weiterentwicklung könnte die aktuelle Implementierung mit der Programmiersprache Golang zu Java portiert werden, sodass eine Integration in das Eclipse IoT Projekt möglich wäre. Diese Integration würde die Reichweite der Lösung erhöhen und dazu beitragen dem Ziel des Eclipse IoT Projektes einen industrieweiten Standard zu schaffen, näher zu kommen. Dazu müsste wie bereits erwähnt eine Migration des aktuellen Projektes vorgenommen werden und sichergestellt werden, dass weiterhin Eclipse Ditto\footnotetext{\url{https://www.eclipse.org/ditto/index.html}} als Digital Twin Provider unterstützt wird, aber zusätzlich auch Eclipse Hono\footnotetext{\url{https://www.eclipse.org/hono/}} als Device Registry und Eclipse Vorto\footnotetext{\url{https://www.eclipse.org/vorto/}} als Strukturregistry verwendet werden kann. 

Eine weitere Möglichkeit der Erweiterung liegt darin, weitere Methoden der Authentifizierung zu untersuchen. Es können unteranderem Ansätze von \citeauthor{fang2020fast} genutzt werden um mithilfe von Machine Learning Verbesserungen bei der Sicherheit großer IoT Anwendungsfälle zu erreichen. \autocite{fang2020fast} So könnte untersucht werden, ob und inwiefern das Konzept umsetzbar ist und welche Auswirkungen es konkret auf die Nutzung digitaler Zwillinge hätte. 

%	Literaturverzeichnis
\clearpage
\ihead{}
\printbibliography[title=Literaturverzeichnis]
\cleardoublepage

% Der Anhang beginnt hier - jedes Kapitel wird alphabetisch aufgezählt. (Anhang A, B usw.)
% \appendix
% \ihead{\appendixname~\thechapter} % Neue Header-Definition


\end{document}
