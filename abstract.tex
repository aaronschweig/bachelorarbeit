\chapter*{Kurzfassung}
\begingroup
\begin{table}[h!]
\setlength\tabcolsep{0pt}
\begin{tabular}{p{3.7cm}p{11.7cm}}
Titel: & \DerTitelDerArbeit \\
Verfasser/in: & \DerAutorDerArbeit \\
Kurs: & \DieKursbezeichnung \\
Ausbildungsstätte: & \DerNameDerFirma\\
\end{tabular}
\end{table}
\endgroup

In folgender Arbeit wird die automatische Erzeugung digitaler Zwillinge thematisiert. Anhand der Defintion eines digitalen Zwillings werden Anforderungen abgeleitet, die ein System mit sich bringen muss, um digitale Zwillinge abzubilden. Dafür werden nachdem Charakteristika digitaler Zwillinge herausgearbeitet wurden, Probleme definiert, die einer automatischen Erstellung im Wege stehen. Zunächst werden Kernkomponentenin Form einer Device Registry und eines Digital Twin Providers identifiziert. Anschließend wird ein Lösungsvorschlag auf Basis der Herausforderungen erstellt, der in der Einführung einer weiteren Komponente resultiert. In einer ausführlichen Analyse werden die identifizierten Probleme bearbeitet, sodass klare funktionelle Anforderungen an die einzelnen Komponenten gestellt werden können. Nachdem eine Analyse der einzelnen Komponenten durchgeführt wurde, wird anhand einer prototypischen Umsetzung die Validität des entwickelten Konzepts überprüft.

