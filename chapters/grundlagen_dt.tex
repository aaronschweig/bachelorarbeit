\chapter{Grundlagen zu Digital Twins}

In folgendem Kapitel wird betrachtet, wie ein \ac{DT} definiert ist und welche Eigenschaften daraus ableitbar sind. Zudem wird versucht eine Unterscheidung und Abgrenzung zu einem Cyberphysischen-System vorzunehmen. Des weiteren werden verschiedene Anwendungsbereich innerhalb der Wirtschaft erläutert, welche alle von einem Einsatz digitaler Zwillinge profitieren würden. Dafür werden einige Beispiele angeführt.

\section{Was sind digitale Zwillinge?}

Innerhalb der Literatur finden sich viele verschiedene Ansätze digitale Zwillinge zu beschreiben. Dabei sticht vorallem heraus, dass oftmals nur bestimmte Teilbereiche sehr detailliert betrachten werden. Dies  führt dazu, dass andere Bereiche, eher weniger detaillreich beschrieben werden. Innerhalb eines Papers von \citeauthor{barricelli2019survey} wird versucht all diese Definitionen zusammenzuführen, mit dem Ziel eine genauere Defintion eines \ac{DT}'s zu erhalten. Dafür werden insgesamt 75 Paper verschiedenster Verläge ausgewertet.\autocite[S. 4, Kapitel 4]{barricelli2019survey} Bei dieser Auswertung ließen sich mehrere Kerneigenschaften digitaler Zwillinge feststellen.

\begin{enumerate}
    \item \enquote{\ac{DT}s can be defined as (physical and/or virtual) machines or computer-based models that are simulating, emulating, mirroring, or \enquote{twinning} the life of a physical entity, which may be an object, a process, a human, or a human-related feature}
    \item \enquote{Each DT is linked to its physical twin through a unique key, identifying the physical twin, and therefore allowing to establish a bijective relationship between the DT and its twin.}
    \item \enquote{A DT is a living, intelligent and evolving model, being the virtual counterpart of a physical entity or process. It follows the lifecycle of its physical twin to monitor, control, and optimize its processes and functions. It continuously predicts future statuses (e.g., defects, damages, failures), and allows simulating and testing novel configurations, in order to preventively apply maintenance operations. More specifically, the twinning process is allowed by the continuous interaction, communication, and synchronization (closed-loop optimization) between the DT, its physical twin and the external, surrounding environment.}
\end{enumerate}

\subsection{\ac{DT}s als virtuelles Abbild eines realen Objekts}

Eine der ersten Kerneigenschaften ist die Abbildung eines realen Objektes durch einen \ac{DT}. Eine der Kernaufgaben eines \ac{DT}s besteht dabei, den realen Zustand des abzubildenden Objektes in einer virtuellen Umgebung zu repräsentieren. Dabei ist es nicht von Bedeutung, wie das unterliegende Objekt aussieht, oder wie groß dessen Umfang ist. So können digitale Zwillinge sowohl von Objekten, als auch von Prozessen, bis hin zu Teilen des menschlichen Organismus gebildet werden. Die grundlegende Aufgabe des digitalen Zwillings bliebt dabei gleich - das Repräsentieren des realen Zustands in einer virtuellen Umgebung.

\subsection{Sicherstellen einer bijektiven Verbindung zwischen \ac{DT} und realem Objekt}

Ein digitaler Zwilling muss eine eindeutige Verbindung zu seinem realen Ursprung haben. Dies kann dadurch erreicht werden, jedem physischen Objekt einen eindeutigen Identifikator zuzuweisen. Dies wird in \citetitle{rios2015product} näher beschrieben. Es werden außerdem verschiedene Möglichkeiten der ID-Vergabe beschrieben. Konventionelle Methoden der Identifizierung stoßen hier an ihre Grenzen. \autocites{barricelli2019survey}{rios2015product}

\subsection{Simulationen und Modellierung auf Basis der gesammelten Daten}

Ein \ac{DT} ist dank der obigen Punkte nun ein exaktes Abbild seines realen Gegenstücks und sowohl der \ac{DT} als auch das physische Objekt sind eindeutig identifizierbar. Auf Basis dieser Grundlage können die gesammelten Daten genutzt werden, um das physische Objekt zu beobachten, evtl. Anomalien festzustellen, usw. Außerdem können zukünfitge Konfigurationsmöglichkeiten einer virtuellen Umgebung getestet werden, bevor diese an kritischen Punkten in einer produktiven Umgebung eingesetzt werden. Diese Tests sind besonders repräsentativ, da so auch Auswirkungen auf die gesamte Umwelt überwacht werden können, ohne dabei Abhängigkeiten mit manuellem Aufwand überprüfen zu müssen. Es kann also mithilfe von \ac{DT}s festgestellt werden, ob und wie stark korrelationen zwischen verscheidenen Eigenschaften bestehen. Ein weiteres Feld, welches sich durch die Nutzung digitaler Zwillinge erschließen lässt ist die \enquote{predicitve Maintanance}. Dabei handelt es sich um ein Konzept, welches vor allem in der Industrie genutzt werden kann. Denn auf Basis der durch den \ac{DT} gesammelten Daten können mithilfe von Datenanalysen, Methoden aus dem Big-Data Bereich und der Verarbeitung durch kunstliche Intelligenz, zukünfitge Zustände des \ac{DT}s abgeleitet werden. Da der \ac{DT} aber nur eine Spiegelung eines realen Objektes ist, die Daten somit von dort stammen, kann angenomen werden, dass ein zukünfitger Zustand des \ac{DT}s auch der zukünfitge Zustand seines realen Gegenstücks ist. Somit kann beispielsweise im Fall einer Maschine eine Wartung geplant und durchgeführt werden, ohne einen tatsächlichen Fehlerfall registrieren zu müssen.

