\chapter{Grundlagen zu Digital Twins}

In folgendem Kapitel wird betrachtet, wie ein \ac{DT} definiert ist und welche Eigenschaften daraus ableitbar sind. Zudem wird versucht eine Unterscheidung und Abgrenzung zu einem Cyberphysischen-System vorzunehmen. Des weiteren werden verschiedene Anwendungsbereich innerhalb der Wirtschaft erläutert, welche alle von einem Einsatz digitaler Zwillinge profitieren würden. Dafür werden einige Beispiele angeführt.

\section{Was sind digitale Zwillinge?}

Innerhalb der Literatur lassen sich verscheidenste Ansätze erkennen, einen \ac{DT} zu beschreiben. Daraus leiten sich auch unterschiedliche Definitionen für einen \ac{DT} ab.

Laut \citeauthor{fuller2020digital} wurde der Begriff \enquote{Digitaler Zwilling} das erste mal von \citeauthor{grieves2014digital} im Jahre 2003 eingeführt, um später einem entsprechenden Whitepaper \citetitle{grieves2014digital} aus dem Jahr \citeyear{grieves2014digital} festgelegt zu werden.\autocite{fuller2020digital}

