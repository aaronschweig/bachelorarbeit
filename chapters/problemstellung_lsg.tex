\chapter{Problemstellung und Lösungsansatz}

Im folgenden Kapitel soll das Problem, welches bei der automatischen Erzeugung digitaler Zwillinge besteht, beschrieben werden. Dafür wird zuerst ein Ablauf dargestellt anhanddessen ein \ac{DT} erstellt wird. Dabei wird dann auf den aktuellen Ist-Zustand eingegangen. Es wird außerdem ien Soll-Zustand definiert, welcher erreicht werden soll, sodass spätere Konzepte daran gemessen werden können. Am Ende dieses Kapitels wird evaluiert, wo die Diskrepanz zwischen dem aktuellen Ist-Zustand und dem Soll-Zustand liegt.

\section{Die automatische Erzeugung digitaler Zwillinge}


\begin{itemize}
    \item Trotz vieler verschiedener Anbieter von DTs werden diese nicht automatisch erzeugt, oder haben nicht den vollen funktionsumfang eines DT.
    \item Der Weg sollte direkt von einem selbstidentifizierenden Sensor stammen, der dann direkt in dem korrespondierenden Framework von DTs einen virtuellen Part bekommt.
\end{itemize}

\section{Ist Zustand}

\begin{itemize}
    \item DTs werden \enquote{manuell} erstellt.
    \item Sensoren haben ein fest definiertes Sensorenset; der DT ist darauf begrenzt $\rightarrow$ muss erweitert werden
    \item Mehr Automatisierung zur Erstellung von DTs unabhängig des verwendeten Schnittstellenformates. Viele verschiedene Austauschformate bringen eine erhebliche Last auf den DT-Provider
    \item 
\end{itemize}